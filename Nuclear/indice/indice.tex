\documentclass[spanish]{article}

\usepackage[spanish,es-tabla]{babel}
\usepackage[margin=1in]{geometry}
\pagestyle{empty}

% file1.pdf: pages  1- 3
% file2.pdf: pages  4- 9
% file3.pdf: pages 10-18

\newcommand{\addsection}[3]{\addtocontents{toc}{\protect\contentsline{section}{\protect\numberline{#1}#2}{#3}}}
\newcommand{\addsubsection}[3]{\addtocontents{toc}{\protect\contentsline{subsection}{\protect\numberline{#1}#2}{#3}}}

\begin{document}


\title{\Huge Informes de Prácticas de Física Nuclear}

\author
{
	Adrián Rivero Fernández%, 49748426R
	\\arivero176@alumno.uned.es
	\\682832933
	\\ Tecnicas Experimentales IV
	\\ Prácticas realizadas los días 4 y 5 de marzo de 2024
	%author 1 works in "institution" and studies for PhD at University (so he is active in both)
	%\and 
	%Name author 2 \\
	%Institution \\
	%Department
	%\\ Tutor: 
}





%\title{Péndulo simple}

%\author{Adrián Rivero Fernández}

\date{\today}

\maketitle






\addsection{}{Caracterización de un Contador Geiger-Müller}{2}
\addsection{}{Estadística Aplicada a Medidas Nucleares}{8}
\addsection{}{Absorción de Partículas Beta}{12}
\addsection{}{Detectores de Centelleo. Absorción de Radiación Gamma}{16}
\addsection{}{Espectrometría de Partículas Alfa y Beta. Absorción de Partículas Alfa}{28}
\addsection{}{Espectroscopía Gamma con detectores de INa(Tl)}{37}
%\addsection{}{Circuitos lineales RC y RL: comportamiento sinusoidal permanente}{30}
%\addsection{}{El diodo}{33}
%\addsection{}{El transistor bipolar}{40}







% Construct contents
%\addsection{1}{File 1-1}{1}
%\addsubsection{1.1}{File 1-1.1}{2}


\tableofcontents

\end{document}