\documentclass[a4paper,12pt,spanish]{article}

\usepackage[spanish,es-tabla]{babel}
\usepackage[margin=1in]{geometry}
\pagestyle{empty}

% file1.pdf: pages  1- 3
% file2.pdf: pages  4- 9
% file3.pdf: pages 10-18

\newcommand{\addsection}[3]{\addtocontents{toc}{\protect\contentsline{section}{\protect\numberline{#1}#2}{#3}}}
\newcommand{\addsubsection}[3]{\addtocontents{toc}{\protect\contentsline{subsection}{\protect\numberline{#1}#2}{#3}}}

\begin{document}


\title{\Huge Informes de prácticas}

\author
{
   Adrián Rivero Fernández, 49748426R
\\arivero176@alumno.uned.es
\\682832933
\\ Tecnicas Experimentales 2
\\ Centro asociado UNED Denia-Benidorm
\\ Prácticas realizadas en el centro asociado de Madrid%author 1 works in "institution" and studies for PhD at University (so he is active in both)
%\and 
%Name author 2 \\
%Institution \\
%Department
\\ Tutor: David Paul del Valle
}



%\title{Péndulo simple}

%\author{Adrián Rivero Fernández}

\date{\today}

\maketitle






\addsection{}{Oscilaciones de torsión. Teorema de Steiner}{2}
\addsection{}{Conservación de la energía mecánica}{14}
\addsection{}{Cuerda vibrante}{25}
\addsection{}{Distribución de campo magnético de una bobina plana}{37}
\addsection{}{Circuitos lineales RC y RL: comportamiento transitorio}{48}
\addsection{}{Circuitos lineales RC y RL: comportamiento sinusoidal permanente}{55}
\addsection{}{El diodo}{62}
\addsection{}{El transistor bipolar}{78}







% Construct contents
%\addsection{1}{File 1-1}{1}
%\addsubsection{1.1}{File 1-1.1}{2}


\tableofcontents

\end{document}