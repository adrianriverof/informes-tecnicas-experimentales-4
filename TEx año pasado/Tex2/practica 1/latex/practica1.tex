%La clase del documento le indica al LaTeX c�mo formatear el contenido,
% que es lo que realmente se indica a continuaci�n, bas�ndose en su
% significado (sem�ntica); las cosas no se representan bas�ndose en un
% sentido est�tico (aunque tambi�n se considera), sino en el papel que
% desempe�an en el escrito
\documentclass[a4paper,12pt,spanish]{article}

%twoside,twocolumn,

%Estos dos paquetes son necesarios para escribir c�modamente los
% los documentos en Español, sin necesidad de utilizar secuencias
% de escape como {\'\i} para poner una "�".
\usepackage[T1]{fontenc}
%\usepackage[latin9]{inputenc}
\usepackage[utf8]{inputenc}

%Para incluir figuras se necesitar�n las macros definidas en este paquete
\usepackage{graphicx}

%El paquete Babel hace que LaTeX seleccione algunos textos preprogramados
% en el idioma pasado como opci�n al documentclass (en este caso, spanish)
%Por ejemplo, la fecha, o los nombres de las figuras o tablas
%(a las tablas las llama cuadros...)
%\usepackage{babel}
\usepackage[spanish,es-tabla]{babel}


%\addto\shorthandsspanish{\spanishdeactivate{~<>}}
%\usepackage{textcomp}

%Este paquete permite incluir y resaltar URLs
\usepackage{url}

\usepackage{enumitem}
%\setlist{nosep}

%Este paquete, cuando se usa pdflatex, formatea el documento a�adiendo
% enlaces internos entre las referencias y los objetos referenciados
% (por ejemplos, figuras, tablas, referencias de la bibliograf�a).
\usepackage[unicode=true, pdfusetitle,
 bookmarks=true,bookmarksnumbered=false,bookmarksopen=false,
 breaklinks=true,pdfborder={0 0 1},backref=false,colorlinks=false]
 {hyperref}

%Este paquete es muy �til, porque permite incluir c�digo escrito
% en varios lenguajes de programaci�n con el formato adecuado con
% s�lo indicar el archivo fuente.
\usepackage{listings}

%Este paquete puede ser �til si se quieren incluir algunos s�mbolos
% especiales en las ecuaciones
%\usepackage{amssymb}


\usepackage{siunitx} %para el sistema internacional
\usepackage[export]{adjustbox}
\usepackage{booktabs} 
 \usepackage{subcaption}
 
 \usepackage{float}
 


%Esta definici�n permite introducir la direcci�n de los autores
% y mostrarla convenientemente
\newcommand{\address}[1]{
\par {\raggedright #1
\vspace{1.4em}
\noindent\par}
}




%esto es para modificar la numeración
\pagenumbering{gobble}
\include{noNumberPage}
\pagenumbering{arabic}
\setcounter{page}{1}

%tutorial de tablas latex: https://manualdelatex.com/tutoriales/tablas

\usepackage{multirow}




%Inicio del documento (hasta que se cierre con \end{document}
\begin{document}



\title{Prueba 1 de Evaluación Continua}
%\address{Mecánica Teórica 2022} 
\author{Adrián Rivero Fernández}
\date{}


\maketitle




\section*{Problema}
%\subsection*{a)}
\textbf{a) Analice el problema del péndulo simple con el método hamiltoniano sin
asumir que el ángulo de giro sea pequeño. Para ello, obtenga el hamiltoniano,
establezca las ecuaciones de Hamilton y resuelva dichas ecuaciones por
cuadraturas.
}
\vspace{\baselineskip}
El péndulo simple es un sistema con un grado de libertad. Escogemos la coordenada generalizada $\theta$ que representa el ángulo del péndulo con la vertical. 

Para un péndulo de longitud $l$ y masa $m$, el lagrangiano vendrá dado por 
\[ \mathcal{L} = T - V
\]
Donde 
\[T = \frac{m}{2} v_m^2\]
\[V = - mgy_m\]

\[x_m = l\sin \theta \Longrightarrow \dot{x_m}=l\dot{\theta}\cos\theta
\]
\[y_m = -l\cos\theta \Longrightarrow \dot{y_m} = l\dot{\theta}\sin\theta
\]

\[v_m^2 = x_m^2+y_m^2 = l^2\dot{\theta}^2
\]


\[ \mathcal{L}(\theta,\dot{\theta}) = \frac{1}{2}ml^2\dot{\theta}^2 + mgl\cos\theta
\]

Y de la expresión del momento conjugado deducimos

\[p_\theta = \frac{\partial\mathcal{L}}{\partial \dot{\theta}} 
= 
ml^2\dot{\theta}
\]
\[p_\theta = ml^2\dot{\theta}  \Longleftrightarrow  
\dot{\theta} = \frac{p_\theta}{ml^2}  \]

El hamiltoniano será por tanto:
\[ H = p_\theta \dot{\theta} - \mathcal{L}
\]
\[ H(\theta,p_\theta) = \frac{p_\theta^2}{2ml^2} - mgl\cos\theta
\]

Y utilizando las ecuaciones de Hamilton obtenemos que

\[\dot{\theta} = \frac{\partial H }{\partial p_\theta } = 
\frac{p_\theta}{ml^2}
\]
\[\dot{p_\theta} =  \frac{\partial H }{\partial \theta }=
-mgl\sin\theta
\]

Podemos resolver por cuadraturas estas ecuaciones obteniendo:

\[ \int  d\theta = \int \frac{p_\theta}{ml^2} dt 
\]
\[ \int dp_\theta = -\int mgl\sin\theta dt
\]


\vspace{\baselineskip}
%\subsection*{b)}
\textbf{b) Con ayuda de los corchetes de Poisson, obtenga la ley de movimiento
	general del péndulo. Deduzca que el movimiento es armónico cuando la
	amplitud del movimiento es suficientemente pequeña.
}






\end{document}



