




\chapter{Efecto Hall en p-Germanium}

\section{Resumen}




\section{Introducción teórica}

\iffalse
En la primera parte de la memoria (en su introducción), y con el objetivo de comprender teóricamente el fenómeno que vamos a estudiar, deberá exponer los fundamentos en los que se basa la práctica. No se trata de copiar únicamente lo que pone en este guión, sino de resumirlo y profundizar un poco más en algún aspecto, pero sin ser extenso.
\fi


\vspace{3\baselineskip}

El efecto Hall se aprecia cuando se hace circular una corriente por una lámina conductora o semiconductora en presencia de un campo magnético. 

Las cargas que circulan se someten a la Fuerza de Lorentz $\vec{F} = q(\vec{v} \times \vec{B})$ y son desplazadas hacia los bordes de la lámina. Esto provoca la aparición de un exceso de carga negativa en uno de los bordes, y de carga positiva en el otro, lo que provoca un campo eléctrico $\vec{E}$, que genera una fuerza $\vec{F}= q \vec{E}$, de misma dirección pero sentido opuesto al campo magnético. 

Las cargas se acumulan y el campo eléctrico crece, hasta que la fuerza eléctrica llega a compensar la magnética. 
Entre los bordes aparece una diferencia de potencial que se llama voltaje Hall.

\begin{equation} \label{eq:vhall}
V_{Hall} = \frac{I B}{nqd}
\end{equation}


\iffalse
\begin{equation}\label{eq:pythagoras}
a^2 + b^2 = c^2 .
\end{equation}


La ecuación \eqref{eq:pythagoras}. A ver cómo queda ¿podré referenciar sin necesidad de un label complejo?

\begin{equation} \label{1.2}
a^2 + b^2 = c^2 .
\end{equation}

Voy a intentar referenciar esta otra \eqref{1.2}
\fi



La polarización dependerá de si las cargas circulantes son positivas o negativas. En los semiconductores, podemos encontrar electrones o huecos, por lo que el signo del potencial permite deducir el tipo de portador. 

La conductividad $\sigma_0$, la movilidad del portador de carga $\mu_{Hall}$ y la concentración de portadores de carga $n$ están relacionados a través de la constante de Hall:

\begin{equation}\label{eq:Rhall}
R_{Hall} = \frac{V_{Hall}}{B} \cdot \frac{d}{I} 
\end{equation}


\begin{equation}\label{eq:muhall}
\mu_{Hall} = R_{Hall}\cdot \sigma_0
\end{equation}

\begin{equation}\label{eq:nconcportadores}
n = \frac{1}{q\cdot R_{Hall}}
\end{equation}



%% Debería expandir un poco más


\section{Descripción del experimento}

%\subsection{Dispositivo experimental}

El montaje que utilizaremos consiste en un soporte vertical con el que posicionamos una lámina conductora de modo que sea afectada por el campo magnético generado por las dos bobinas. La lámina tiene un sistema para medir distintas variables, que son procesadas e introducidas en un software informático en el ordenador, desde el que podemos controlar todo el equipo. El equipo está alimentado por una fuente de corriente.

\begin{figure}[H]
	\centering
	\includegraphics[width=0.7\linewidth]{"../1-efecto hall/IMG_20240306_144550"}
	\caption{Montaje Experimental para el efecto Hall}
	\label{fig:img20240306144550}
\end{figure}




%\subsection{Toma de datos}
%Creo que no quiere que le expliquemos todo "esa información la tenemos todos"


\section{Resultados y análisis}

\subsection{Experimento 1}

Representamos V Hall en mV vs. intensidad de corriente lp en mA.

Con el ajuste lineal sacamos la constante de proporcionalidad $\alpha$ de $V = \alpha I$

Ahora sabiendo que nuestra lámina tiene d = 1mm, calculamos la densidad de portadores $n$ despejando de la expresión $\alpha $ y \eqref{eq:Rhall}

Para la propagación de errores necesitaría el valor exacto de B (y saber del todo su maldito error)


\subsection{Experimento 2}

Representar V Hall en mV vs. intensidad de campo magnético B en mT.
Nos quedará una relación lineal\\
V Hall = b · B

Con ella podemos expresar la constante de hall R Hall (sacando expresiones)

La intensidad fue fijada (necesito su dato exacto). 

Recalcular n, calcular mu hall, sigma con una formula nueva que nos dan. Hay unas cuantas medidas que no nos dan el error, y luego piden temperatura del exp 5. 

Finalmente comparar con la de la literatura (que habrá que buscarse la vida en la literatura)


\subsection{Experimento 3}

con las medidas de V las procesamos (conociendo el valor de I) y sacamos la resistencia, nos piden que representemos el cambio relativo de resistencia frente al campo magnetico B.

Esta representación debe salir que la R aumenta mucho con el campo B (no es linea recta). 

Tengo que explicar este resultado. Tratar de ajustar el comportamiento a un polinomio (algún tipo de ajuste polinómico). Tengo que explicar el resultado, me dan la pista de que es un fenómeno utilizado en escritura y lectura de discos duros (asumo que HDDs)



\subsection{Experimento 4}

Representar V Hall en función de la temperatura en ºC.

Obtendremos que el potencial decrece con la temperatura. 

Tengo que explicar a qué se debe esto. Me dan la pista de que tenga en cuenta el exp 5, que al crear nuevos portadores con la temperatura, se crean también más huecos y electrones. Cuidado porque tengo que pillar bibliografía para esto.


\subsection{Experimento 5}




\section{Conclusiones}


\section{Bibliografía}






